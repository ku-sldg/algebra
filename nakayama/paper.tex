\RequirePackage[l2tabu,orthodox]{nag}
\documentclass{article}
\usepackage{amsthm}
\usepackage{amssymb}
\usepackage{mathtools}
\usepackage{minted}

\usepackage[colorlinks=true,pagebackref,hyperindex,citecolor=myblue,linkcolor=mygreen,urlcolor=mygreen]{hyperref}
\usepackage[top=1in, bottom=1in, left=1.5in, right=1.5in]{geometry}

\usepackage{amsmath}
\usepackage{amsfonts}
\usepackage{amssymb}

\usepackage{color}
\usepackage[dvipsnames]{xcolor}
%\definecolor{myblue}{rgb}{0.16, 0.32, 0.75}
\definecolor{myblue}{rgb}{0.0, 0.45, 0.73}
\definecolor{mygreen}{rgb}{0.05, 0.5, 0.06}

\usepackage{mathrsfs}
\usepackage[mathcal]{euscript}
\usepackage{mathtools}

\usepackage{enumitem}

\usepackage{float}
\usepackage[center]{caption}
\usepackage{graphicx}
\graphicspath{ {./images/} }

%\usepackage{setspace}
%\singlespacing
%\onehalfspacing
%\doublespacing
%\setstretch{1.1}
\parskip 0.1cm


%%%%%%%%%%%%%%%%%% Clever Ref %%%%%%%%%%%%%%%%%%
\usepackage{cleveref}
\crefname{equation}{Eq.}{Eqs.}
\crefname{theorem}{Theorem}{Theorems}
\crefname{lemma}{Lemma}{Lemmata}
\crefname{corollary}{Corollary}{Corollaries}
\crefname{proposition}{Proposition}{Propositions}
\crefname{definition}{Definition}{Definitions}
\crefname{rem}{Remark}{Remarks}
\crefname{example}{Example}{Examples}
\crefname{conventions}{Conventions}{Conventions}
\crefname{setup}{Setup}{Setup}
\crefname{question}{Question}{Question}
\crefname{figure}{Figure}{Figures}


%THEOREM ENVIRONMENTS
\newtheorem{theorem}{Theorem}[section]
\newtheorem{lemma}[theorem]{Lemma}
\newtheorem{corollary}[theorem]{Corollary}
\newtheorem{proposition}[theorem]{Proposition}

\newtheorem*{nak*}{Nakayama's Lemma}


%DEFINITION STYLE ENVIRONMENTS
\theoremstyle{definition}
\newtheorem{definition}[theorem]{Definition}
\newtheorem{definition-theorem}[theorem]{Definition/Theorem}
\newtheorem{setup}[theorem]{Setup}
\newtheorem{notation}[theorem]{Notation}
\newtheorem{conventions}[theorem]{Conventions}
\newtheorem{example}[theorem]{Example}

%REMARK STYLE ENVIRONMENTS 
\theoremstyle{remark}
\newtheorem{remark}[theorem]{Remark}
\newtheorem{convention}{Convention}

\numberwithin{equation}{section} %Can replace {subsection} with {theorem} if you want
\numberwithin{figure}{subsection} %Can replace {subsection} with {theorem} if you want

%%%%%%%%

%\newcommand{\pagebreak}{\newpage}
\renewcommand{\pagebreak}{ }


\mathchardef\ordinarycolon\mathcode`\:
\mathcode`\:=\string"8000
\begingroup \catcode`\:=\active
  \gdef:{\mathrel{\mathop\ordinarycolon}}
\endgroup

%%%%%%%%%%%%%%%%%% To-Do %%%%%%%%%%%%%%%%%%
\usepackage[textwidth=2.5 cm,textsize=small,shadow,
%disable
%%option disable removes the notes
]{todonotes}
\newcommand{\drew}[2][]
{\todo[linecolor=green,backgroundcolor=green!10,caption={}, #1]{#2}}
\newcommand{\emily}[2][]
{\todo[linecolor=gray,backgroundcolor=gray!20,caption={}, #1]{#2}}
\newcommand{\perry}[2][]
{\todo[linecolor=blue,backgroundcolor=blue!20,caption={}, #1]{#2}}


\begin{document}
\title{ 
Formalizing Commutative Algebra in Coq: \\
Nakayama's Lemma\thanks{Source code for this work is available on the following
site: \url{https://github.com/ku-sldg/algebra}}
}

\date{}

\author{ 
	Andrew Cousino \\ {\ttfamily acousino@ku.edu}
	\and 
	Emily E.\ Witt\thanks{Witt acknowledges support from NSF CAREER Award
    DMS-1945611 and the 2022-23 Ruth I.\, Michler Memorial Prize from the
    Association for Women in Mathematics.} \\ {\ttfamily witt@ku.edu}
  \and 
	Perry Alexander \\ {\ttfamily palexand@ku.edu}
}	

\maketitle	

\vspace{-1cm}

{\large
\begin{center}
	Institute for Information Sciences \\
	The University of Kansas \\
	Lawrence, KS 66045
\end{center}
}

\todo[inline]{To do:  Add to/alter author list, funding acknowledgments, title,
  and/or abstract as needed}

\begin{abstract}
  We describe our formal proof of Nakayama's Lemma, a fundamental theorem in
  the mathematical field of commutative algebra. The statement and proof of
  this result involve several commutative-algebraic structures including
  commutative rings, ideals of these rings, and modules over them, and we also
  explain our process of formalizing these structures.
\end{abstract}

\noindent \textbf{Keywords:}
Formalization of Mathematics,
Formal Proof,
Commutative Algebra,
Commutative Ring,
Local Ring,
Ideal,
Module over a Ring,
Finitely Generated Module. 

\section{Introduction}
The mathematical field of \emph{commutative algebra} stems from the study of
solutions to polynomial equations. Research in the field now centers around
\emph{commutative rings}---rings in which order does not affect multiplication,
i.e., $x \cdot y = y \cdot x$ for any ring elements $x$ and $y$---and
fundamental algebraic objects associated to them:  \emph{ideals} of these
rings, and \emph{modules} over them. Commutative algebra has deep connections
with other areas of theoretical mathematics, including number theory and
algebraic geometry. 

Commutative algebra also has broad applications to science and technology. For
instance, it has been integral to advances in robotics \cite{cox-little-oshea},
and has helped form our current understanding of the human genome
\cite{genetic-algebra}. The commutative-algebraic notion of a Gr\"{o}bner
basis, a special type of generating set for an ideal in a ring of polynomials,
has become a fundamental computational tool in coding theory and cryptography
(e.g., see  \cite{grobner-bases-cryptography}). A implementation of
Buchberger's algorithm \cite{buchberger} for determining Gr\"obner bases of
ideals in polynomial rings has been proved correct within the proof assistant
Coq \cite{the_coq_development_team_2019_3476303,thery-buchberger}, and an
integrated formal development of the algorithm in Coq has also been carried out
\cite{persson2001integrated} (see also \cite{grobner-type-theory}). 

Our goal is to newly formalize theoretical, rather than computational,
commutative algebra in Coq. We formally prove \emph{Nakayama's Lemma}
\cite{nakayama-1951, azumaya}, an essential result in the field. In doing so,
we formalize algebraic structures that are fundamental to higher-level algebra,
such as \emph{local rings} and \emph{modules over commutative rings}, and
\emph{quotient rings and modules}. Rather than build upon some of the basic
objects from abstract algebra, such as groups and rings, that have been
formalized in Coq, e.g., in the \emph{Mathematical Components Library}
\cite{mathcomp}, we start from scratch.
\todo{To do: Verify whether MathComp only formalized finite algebraic
  structures. Drew is about as certain as he can be that this is the case.}
The theory, including the formalization of all algebraic structures, makes up
approximately $100$ kB and 3300 lines of code.
 
The notion of a module over a ring is an extension of the linear-algebraic
notion of a vector space over a field, ubiquitous in mathematics and its
applications. Less frequently referred to as the \emph{Krull-Azumaya theorem}
\footnote{Hideyuki Matsumura explains in his text \emph{Commutative Algebra}
  \cite{matsumura}: ``{This simple but important lemma is due to T.\ Nakayama,
  G.\ Azumaya, and W.\ Krull. Priority is obscure, and although it is usually
  called the Lemma of Nakayama, late Prof.\  Nakayama did not like the
  name.''}}
\,\cite{nagata}, Nakayama's Lemma describes one way that a finitely generated
module over an arbitrary commutative ring acts like a vector space over a
field. True to the convention that ``lemma'' often refers to a result serving
as a stepping stone toward another goal, Nakayama's Lemma is applied widely
throughout the field, and the result is typically introduced in a first
graduate course in commutative algebra \cite{atiyah-macdonald, matsumura,
eisenbud}.

% It is widely applied throughout algebra. %, true to. However, Nakayama's lemma 
% is also an important result in its own right, and can be thought of as one of
% the fundamental theorems of 
% commutative algebra, not unlike the fundamental theorems of calculus
% describing the inverse relationship between derivatives and integrals or the
% fundamental theorem of arithmetic that makes precise the notion of unique
% factorization.
% , or the fundamental theorem of algebra (or \emph{d'Alembert's theorem}) on
% the roots of complex polynomials.

\section{Mathematical Basis and Motivation}
\subsection{The Fundamental Algebraic Structures}
Here, we give a brief description of the major mathematical structures from
commutative algebra that are relevant to Nakayama's Lemma.

\paragraph{Commutative rings.}
In abstract algebra, the quintessential example of a commutative ring is the
set of integers
\[\mathbb{Z} = \{ \ldots, -3, -2, -1, 0, 1, 2, 3, \ldots \}.\]
using the natural definitions of addition and multiplication.

Adding two integers produces another, and the associative and commutative laws
hold for addition. The integers form an \emph{abelian group} under addition
since $0 \in \mathbb{Z}$ is the \emph{additive identity} in the sense that
adding zero has no effect on any integer, and given any integer $n$, the
integer $-n$ is its \emph{additive inverse} in the sense that the sum of $n$
and $-n$ is the additive identity $0$. 

The set of integers also forms a \emph{ring} due to its properties of
multiplication. It is closed under this binary operation, which satisfies
associativity, and the distributive law governing the compatibility of addition
and multiplication holds.  We require rings to contain a \emph{multiplicative identity},
and $1 \in \mathbb{Z}$ is such an element since $n \in \mathbb{Z}$ one has
$n \cdot 1 = 1 \cdot n = n$. 
Even more, the integers form a
\emph{commutative ring} since $n \cdot m = m \cdot n$ for all integers $n$ and
$m$. 

In general, a commutative ring is a set $R$ with two binary operations, which
we call \emph{addition} and \emph{mutiplication}, typically denoted $\cdot$ and
$+$, respectively. As motivated by the properties of the ring of integers,
addition, $R$ must be an abelian group, multiplication must be associative, $R$
must have a multiplicative identity, and the distributive law must hold, i.e.,
for all $r, s, t \in R$, $(r+s)\cdot t = r \cdot t + s \cdot t$ and
$r \cdot (s+t) = r \cdot s + r \cdot t$.

Other familiar examples of commutative rings include the integers modulo a
fixed integer $n>0$, fields---commutative rings in which every nonzero element
has a multiplicative inverse---such as the rings of rational numbers, real
numbers, and complex numbers, and rings of polynomials in a variable $x$ with
integer coefficients, or with coefficients in a field.


\paragraph{Ideals of commutative rings.}
The concept of an {ideal} of a ring can be thought of as an extension of the
notion of an integer $n$ in the ring of integers $\mathbb{Z}$. An \emph{ideal}
of commutative ring $R$ is a subset $I$ of $R$ that is itself an abelian group
under addition, which also satisfies the following ``absorption'' property:
Given any element $a$ of $I$, the product  $x \cdot a$ is again in $I$ for any
ring element $x \in R$. 

One can verify that given any integer $n$, the set $n\mathbb{Z}$ of its
multiples forms an ideal of $\mathbb{Z}$. For instance, $2 \mathbb{Z}$
consists of all even numbers, and is an abelian group under addition: the sum
of two even numbers is even, the additive identity $0$ is even, and the
negative of an even number is even. Moreover, the absorption property holds
since the product of any integer and an even number is again even. In fact,
every ideal of the ring of integers has this form $n\mathbb{Z}$ for some
integer $n$, though ideals in general commutative rings can have more
complicated properties. 

Since every integer $n$ can be written as $1 \cdot n$, the ideal $1 \mathbb{Z}$
is the entire ring $\mathbb{Z}$. One can see that given a commutative ring
$R$ itself satisfies the axioms required to be an ideal of $R$. We call an
ideal $I$ of $R$ \emph{proper} if it is strictly contained in $R$. The
\emph{zero ideal} consisting solely of its additive identity is a proper ideal
of any commutative ring. 

A \emph{prime ideal} of a commutative ring is a proper ideal $I$ with the following property:  
If the product $x \cdot y$ of ring elements $x$ and $y$ is in $I$, then $x \in I$ or $y \in I$.
The naming convention is motivated by the ring of integers, where the prime ideals are precisely those 
of the form $p\mathbb{Z}$, where $p$ is a prime number, along with the zero ideal. 

A \emph{maximal ideal} of a commutative ring is a proper ideal that is maximal
with respect to inclusion, i.e., no other proper ideal strictly contains it. 
Returning to our example of the ring of integers,
$6 \mathbb{Z} \subsetneq 2 \mathbb{Z}$ since every multiple of $6$ is even, so
$6 \mathbb{Z}$ is not a maximal ideal of $\mathbb{Z}$. However, no proper ideal
$I$ contains $2 \mathbb{Z}$: If $2\mathbb{Z}\subsetneq I\subsetneq \mathbb{Z}$,
then $I$  would necessarily contain an odd number $n$. Writing $n=2k+1$ for
some integer $k$, we notice that since $-2k$ is in $2\mathbb{Z}$, it is also an
element of the larger set $I$, and since $I$ is an abelian group under
addition, $(2k+1) + (-2k) = 1$ is also in the ideal $I$. However, in this case,
every integer $n = n \cdot 1$ is in $I$ by absorption, so $I = \mathbb{Z}$ is
not a proper ideal, a contradiction. 

In fact, $3 \mathbb{Z}$ is the only other maximal ideal of $\mathbb{Z}$
containing $6 \mathbb{Z}$, and in general, the prime ideals in the ring of
integers besides the zero ideal are  $p \mathbb{Z}$, where $p$
a prime number. 
It is not a coincidence that every maximal ideal of the ring of integers is also a prime 
ideal; the analogous statement can be proved in arbitrary commutative rings. 


\paragraph{Local rings.}
A commutative ring is \emph{local} if it has exactly one maximal ideal. 
Every field is local since the only proper ideal of a field is the zero ideal,
though by our observations above, the ring of integers is not local. 
However, the set of all rational numbers that can be written with an odd
denominator does form a subring of all rational numbers, and its unique maximal
ideal consists of the elements with even numerator; in fact, this ring is the
so-called \emph{localization} of $\mathbb{Z}$ at the maximal ideal
$2\mathbb{Z}$. The ring of integers modulo $n>1$ is local if and only if $n$ is
a power of a prime number $p$, in which case the unique maximal ideal consists
of all multiples of $p$. 

The ring of polynomials over a field $F$ in a variable $x$ is not local; in
fact, given any irreducible polynomial $f(x)$, the set of its multiples is a
maximal ideal of the polynomial ring $F[x]$. On the other hand, the set of all
formal power series  in $x$ over $F$ is a local ring; its maximal ideal
consists of the power series with no constant term. 

% Hence the ring of integers is not local, but the ring of integers modulo an
% integer $n>1$, often denoted $\mathbb{Z}/n\mathbb{Z}$ or $\mathbb{Z}_n$, are
% local 

\paragraph{Modules over commutative rings.}
Let  $R$ be a commutative ring. 
A \emph{module over $R$}, or \emph{$R$-module}, is an abelian group $M$ under a
binary operation $+$, and a {scalar multiplication} $R \times M \to M$ denoted
$\cdot$, satisfying the following properties for all $r, s \in R$
and $u, v \in M$.

\begin{enumerate}[leftmargin=5cm,topsep=0cm,itemsep=0cm]
  \item  $r \cdot (u + v)  = r \cdot u + r \cdot v$
  \item $(r+s) \cdot u = r \cdot u + s \cdot u$
  \item $(rs) \cdot u = r \cdot (s \cdot u)$
  \item $1 \cdot u = u$ 
\end{enumerate}

From this definition, one can see that a module over a field $F$ is precisely
an $F$-vector space, so the notion of a module over an arbitrary commutative
ring extends that of a vector space over a field. Finitely generated vector
spaces form the foundation for matrix algebra, and the extension of this notion
to module theory is needed to state Nakayama's Lemma. We call an $R$-module
$M$ \emph{finitely generated} if there exist a fixed finite number of elements
$u_1, \ldots, u_n$ of $M$ with the following property:  Given any  $w \in M$,
there exist $r_1, \ldots, r_n \in R$ for which  
\[w = r_1 u_1 + r_1 u_2 + \cdots + r_n u_n\text{.}\]
The set $\{u_1, \ldots, u_n\}$ is called a \emph{generating set} for the $M$ as
an $R$-module. 

When $R = F$ is a field and $M = V$ is a finite-dimensional vector space over
$F$, one can choose $u_1, \ldots, u_n$ to be a basis for $V$, i.e.,
$n = \dim V$. In this case, the choice of scalar coefficients in the expression
above for $w \in V$ is unique. When $R$ is not a field, however, such an
expression is typically not unique. 


\subsection{Nakayama's Lemma, Informal Statement}

In order to state Nakayama's Lemma, we first explain some notation:  If $I$ is
an ideal of a commutative ring $R$ and $M$ is an $R$-module, then $IM$ is the
set of elements of the form $a_1 u_1 + a_2 u_2 + \cdots + a_k u_k$, where, for
some positive integer $k$, $a_1, \ldots, a_k \in I$ and
$u_1, \ldots, u_k \in M$. Notice that due to the absorption property of ideals,
$I M$ is an $R$-module contained in $M$. 	

If an $R$-module $M$ consists of only one element, this element must be its
additive identity $0$ as an abelian group under addition. The notation $M=0$
means that we are in this situation. 

\begin{nak*}
Let $R$ be a commutative local ring, and let $\mathfrak{m}$ denote its unique
maximal ideal. If $M$ is a finitely generated $R$-module and
$M = \mathfrak{m} M$, then $M = 0$. 
\end{nak*}

When $R=F$ is a field, its unique maximal ideal is the zero ideal, and given
any vector space $M=V$ over $F$, the only linear combination of vectors with
coefficients in the zero ideal is the zero vector. Hence in this special case,
the hypothesis that  $M = \mathfrak{m} M$ is equivalent to the conclusion that
$M=0$. Hence Nakayama's Lemma describes one way that finitely generated modules
over commutative local rings are similar to vector spaces. 

In general, the quotient $R/\mathfrak{m}$ of a local ring modulo its maximal
ideal $\mathfrak{m}$ is a field, and the quotient of a module $M$ modulo the
submodule $\mathfrak{m}M$ is an $R/\mathfrak{m}$-module, i.e.,
$M/\mathfrak{m}M$ is a vector space over $R/\mathfrak{m}$. Nakayama's Lemma
implies that if $M$ is finitely generated, then bases for $M/\mathfrak{m}M$
corresponds, via lifting, to minimal sets of generators of $M$. 

We point out that there are alternate statements of Nakayama's Lemma that do
not require the hypothesis that $R$ must be local. One can replace the unique
maximal ideal with the Jacobson radical of the ring, which is the intersection
of all maximal ideals. Alternatively, $I$ is an arbitrary proper ideal of a
commutative ring $R$ and $M$ is a finitely generated $R$-module for which
$M=I M$, then this ensures the existence of a ring element $r$ congruent to
$1$ modulo $I$ such that $r M = 0$, i.e., $r u = 0$ for every $u \in M$. 

\section{Formalization}
We start by describing our process of formalizing the required algebraic
structures detailed in the previous section. Then, with that in hand, we move
on to the formal proof of Nakayama's Lemma.

\subsection{Our Algebraic Hierarchy}
\begin{figure}[t]
  \caption{The hierarchy of our algebraic structures}
  \includegraphics[width=.4\textwidth]{algebraic-structures.pdf}
  \centering
\end{figure}

Our foundation begins by defining a semigroup class, which declares a binary
operation to be associative. From here, we build up through monoids, which
introduce identities, to groups, which introduce inverses. Note the double
equals ``$==$'' appearing in these definitions is notation for an arbitrary
equivalence relation over the group's carrier set, which acts as equality.

\begin{minted}{coq}
  Infix "==" := equiv (at level 60, no associativity).
  Class Semigroup := {
    semigroup_assoc:
      forall (a b c: Carrier),
        a <o> b <o> c == a <o> (b <o> c);
  }.
  Class Monoid := {
    monoid_semigroup :> Semigroup equiv op;
    monoid_ident_l:
      forall (a: Carrier), ident <o> a == a;
    monoid_ident_r:
      forall (a: Carrier), a <o> ident == a;
  }.
  Class Group := {
    group_monoid :> Monoid equiv op ident;
    group_inv_l:
      forall (a: Carrier), inv a <o> a == ident;
    group_inv_r:
      forall (a: Carrier), a <o> inv a == ident;
  }.
\end{minted}
Lines such as  ``\verb|monoid_semigroup :> Semigroup equiv op;|'' simply coerce
the monoid typeclass into a semigroup.

%While we found later on that we did not need quotients, 
While in the end, our formal proof
does not require calling upon 
 quotients of algebraic
structures, quotient rings and quotient modules are fundamental to commutative
algebra, and one can use them to construct alternate proofs of Nakayama's
Lemma. It is worth pointing out that we have formalized quotients of algebraic
objects in Coq using typeclasses, which appear to work rather nicely. 

An algebraic quotient is, roughly, the set of equivalence classes of an
algebraic structure with respect to an equivalence relation on its elements,
for which the set of equivalence classes inherit the same kind of algebraic
structure. For example, consider the quotient of a group modulo a subgroup,
i.e., a subset of elements of the group that it itself a group under the group
operations. Under equivalence relation on the group, every element of the
subgroup must be in the same equivalence class as the identity. With \texttt{P}
the predicate for the subgroup, there are two ways to make an equivalence
relation from this description.

\begin{minted}{coq}
  Definition left_congru (a b: Carrier) :=
    P (inv a <o> b).
  Definition right_congru (a b: Carrier) :=
    P (a <o> inv b).
\end{minted}

When these two relations coincide, then we can prove that this common
equivalence relation actually preserves the group structure. Subgroups which
have this property are called \emph{normal subgroups}.

\begin{minted}{coq}
  Let normal_subgroup_congru_coincide :=
    forall (a b: Carrier),
      left_congru op inv P a b <->
      right_congru op inv P a b.

  Theorem quotient_normal_subgroup_group:
    normal_subgroup_congru_coincide ->
    Group (left_congru op inv P) op ident inv.
\end{minted}

The importance of quotients in commutative algebra motivates our use of
equivalence relations to define the components of a group structure. If one
were to instead use the regular Leibniz equality, it would be difficult to
identify a quotient group and another group. However, by  defining a group in
terms of an arbitrary equivalence relation, in our theory a quotient group is 
simply defined as a group, but under an equivalence relation that is not the usual 
equality. 
Not much is lost due to Coq's rewrite tactics for setoids--types equipped with an equivalence relation--
can still be called upon.

Moving onward, rings form the next step in our algebraic hierarchy; a ring has two binary
operations: addition, which must be commutative, and multiplication, which need
not be commutative in general. Next, we formalized the definition of a commutative
ring, further requiring commutativity of multiplication, as well as a
multiplicative identity. At this point, we formalized the notion of an ideal of
a commutative ring, a subgroup of the ring under addition that satisfies the
absorption property under multiplication, i.e., \(r a\) is in the ideal for
every element \(a\) of the ideal, and every element \(r\) of the commutative
ring. We also used this to formalize the notion of a quotient ring $R/I$, where
$I$ is an ideal of a commutative ring $R$.


Next, we formalized the definition of a  prime ideal, and then 
moved on to do the same for the notion of a maximal ideal, a proper ideal 
that is maximal with respect to inclusion. Below is the definition in Coq, which
uses \texttt{P} as the predicate for the ideal.

\begin{minted}{coq}
  Definition maximal_ideal :=
    exists (r: Carrier), (not (P r) /\
      forall (Q: Carrier -> Prop)
          (Q_proper: Proper (equiv ==> iff) Q)
          (Q_ideal: Ideal add zero minus mul Q),
        (forall (r: Carrier), P r -> Q r) ->
        (forall (r: Carrier), Q r) \/
          (forall (r: Carrier), Q r -> P r)).
\end{minted}

We then were able to take advantage of the definition of a maximal ideal
to formally define a local
ring, i.e., a commutative ring with a single maximal ideal.
 
\begin{minted}{coq}
  Definition local_ring :=
    exists (P: Carrier -> Prop)
        (P_proper: Proper (equiv ==> iff) P)
        (P_ideal: Ideal add zero minus mul P),
      maximal_ideal P /\
      (forall (Q: Carrier -> Prop)
          (Q_proper: Proper (equiv ==> iff) Q)
          (Q_ideal: Ideal add zero minus mul Q),
        maximal_ideal Q -> forall (r: Carrier), P r <-> Q r).
\end{minted}

%\emily[inline]{Emily:  Unfortunately, I think we need to update naming
%  conventions since in mathematics, ``vector'' is reserved for an element of a
%  vector space over a field, rather than an element of an arbitrary module.  So
%  this could cause confusion.  Similar with ``basis,'' which is especially
%  tricking since two different generating sets of a finitely generated module
%  can have different sizes, but all bases of a vector space have the same
%  cardinality, and an element of a module can typically be written as a scalar
%  combination in multiple ways, while bases are linearly independent.  I've
%  proposed a rewrite of the blue paragraph right above it.  Drew, can you
%  double check it for accuracy with respect to the code? 
%}
%\emily[inline]{We'll also need to update the code.  I think {\ttfamily vectors}
%  should be something like {\ttfamily scalar-combinations}, and
%  {\ttfamily basis} something like {\ttfamily generating-set}. 
%}

At this point we formalized the definition of a module over a commutative ring,
the commutative-algebraic generalization of the notion of a vector space over a
field. Nakayama's Lemma is a statement about finitely generated modules, and
hence we must formalize the notion of a scalar combination of a finite
collection of elements, $u_1, \ldots, u_n$, of an $R$-module $M$, i.e.,
expressions of the form $r_1 u_1 + r_1 u_2 + \cdots + r_n u_n$, where each
$r_i \in R$.

In our formalization of scalar combinations, we use  ``list'' to mean
length-parameterized lists; since we don't use the simpler kind of lists, there
are no name collisions. In our code, \texttt{M} is the type of module elements,
\texttt{R} is the type of ring elements, act as coefficients, and
\texttt{t A n} is a list whose elements are of type \texttt{A} and whose length
is \texttt{n}.



\begin{minted}{coq}
  Definition finitely_generated {n: nat}(basis: t M n) :=
    forall (vector: M),
      exists (coeffs: t R n),
        vector =M= linear_combin coeffs basis.
\end{minted}

Next, given an ideal \(I\) of a commutative ring $R$ and and an $R$-module
\(M\),  we defined the submodule $IM$, the set consisting of all scalar
combinations of elements of \(M\) whose coefficients are in  \(I\). We
represented this in Coq as a predicate over \(M\).

\begin{minted}{coq}
  Context (P: R -> Prop).
  Context {P_proper: Proper (Requiv ==> iff) P}.
  Context {P_ideal: Ideal Radd Rzero Rminus Rmul P}.
  
  Definition ideal_module (x: M): Prop :=
    exists (n: nat)(coeffs: t R n)(vectors: t M n),
      Forall P coeffs /\
      x =M= linear_combin Madd Mzero action coeffs vectors.
\end{minted}

The use of  ``\verb|Forall P coeffs|'' ensures that every element of the
coefficient list \texttt{coeffs} satisfies the predicate \texttt{P}.  

\subsection{Constructing the Formal Proof}
Beyond formalizing the relevant structures from higher algebra, we also
formally establish some basic theory. For instance, a \emph{unit} of a
commutative ring $R$ is an element $x \in R$ with a multiplicative inverse,
i.e., an element $x^{-1} \in R$ for which $x \cdot x^{-1}$ is the
multiplicative identity $1 \in R$.  In fact, it $x$ is a unit, it has a unique  inverse. 
 
We formally proved that an ideal $I$ of a commutative ring $R$ that contains a
unit $x$, then $I$ must be the trivial ideal, i.e., the entire ring.  The
informal logic is as follows:  By the absorption property, since $x \in I$, we
have that $x \cdot x^{-1} = 1 \in I$.  Hence for every element $r$ of $R$,
$r = r \cdot 1$ is also in $I$, i.e., $I=R$. 

Every non-unit element of a commutative ring is contained in some maximal
ideal. This fact relies on the Axiom of Choice. The following standard informal
proof calls upon Zorn's lemma, which is equivalent to the Axiom of Choice, and
says that given a partially ordered set $S$, if every chain in $S$ has an upper
bound, then $S$ must have at least one maximal element. 

\begin{quote}
    Set \(I_{1}\) to be the principal ideal generated by an element \(x\) of a
    commutative ring $R$, i.e. the smallest ideal containing the element $x$,
    which consists of all $R$-multiples of $x$.
   
    If \(I_{1}\) is not a maximal ideal, then there exists a strictly larger
    proper ideal \(I_{2}\) of $R$, i.e.,
    $x\in I_{1}\subsetneq I_{2} \subsetneq R$. Moreover, if  \(I_{2}\) is not
    maximal, then there exists a strictly larger proper ideal
    $I_{3}$ containing $I_2$.
    
    Continuing this process, if it terminates at some step, we have found a
    maximal ideal, and if not, one obtains an infinite chain of ideals
    containing \(x\).
    \[x\in I_{1}\subsetneq I_{2}\subsetneq I_{3}\subsetneq\cdots\subsetneq R\]
    
    It is straightforward to verify that the increasing union of all $I_{k}$, $k \geq 1$, 
    is by definition an ideal of $R$, and certainly
    contains \(x\).  Hence by Zorn's lemma, since every ascending chain of ideals
    containing $x$ with respect to inclusion has its union as an upper bound,
    there exists a maximal ideal of \(R\) containing \(x\).
\end{quote}

This argument has potentially infinitely many steps, and chose to avoid this
issue by including an axiom that in any non-unit \(x\) of a ring is contained in some
maximal ideal.

\begin{minted}{coq}
Axiom comm_ring_nonunit_maximal_ideal:
  forall (x: Carrier),
    ~ is_unit equiv mul one x ->
    exists (P: Carrier -> Prop)(P_proper: Proper (equiv ==> iff) P)
    	(P_ideal: Ideal add zero minus mul P),
      P x /\ maximal_ideal P.
\end{minted} 

We also used classical logic to prove that \(1 - x\) is a unit whenever \(x\) is
an element of a local ring that is not a unit. The proof is completed by way of contradiction, and uses the rule that  
 \(\neg\neg P\rightarrow
P\). 

To start describing our proof, we formalize the statement of Nakayama's Lemma in Coq.

\begin{nak*}
Let $R$ be a commutative local ring, and let $\mathfrak{m}$ denote its maximal
ideal. Suppose that $M$ is a finitely generated $R$-module. If
$M = \mathfrak{m} M$, then $M = 0$, i.e., $M$ must be the $R$-module containing
only one element, its identity as an additive abelian group. 
\end{nak*}

\begin{minted}{coq}
Theorem nakayama:
  forall {n: nat}(basis: t M n),
    finitely_generated Mequiv Madd Mzero action basis ->
  (forall a: M, ideal_module_pred a) ->
  forall a: M, a =M= Mzero.
\end{minted}

To use as a building block in the formal proof of Nakayama's Lemma, we first formally stated and proved a lemma that gives a concrete description of the elements of the submodule $\mathfrak{m} M$  appearing in the statement of the the theorem.  
This lemma applies more generally, to any finitely generated module $M$ over a (not necessarily local) commutative ring, and any submodule of the form $IM$, for $I$ an arbitrary ideal. 
By definition, $IM$ consists of scalar combinations of elements of $M$ with coefficients in $I$. 
The lemma states that the elements from $M$ appearing in the expression can be chosen to be from any 
 fixed finite generating set for $M$. 
In other words, given any generating set  $u_{1},\dots, u_{n} \in M$  of a  finitely generated $R$-module $M$, every element $x$ of $I M$ can be written as a scalar combination $a_1 \cdot u_1 + a_2  \cdot u_2 + \cdots + a_n  \cdot u_n$, where each $a_i \in I$. 

\begin{minted}{coq}
Lemma module_fin_gen_ideal_module:
  forall {n: nat}(basis: t M n),
    finitely_generated Mequiv Madd Mzero action basis ->
    forall {m: nat}(coeffs: t R m)(vectors: t M m),
      Forall P coeffs ->
      exists (coeffs': t R n),
        linear_combin Madd Mzero action coeffs vectors =M=
          linear_combin Madd Mzero action coeffs' basis /\
        Forall P coeffs'.
\end{minted}        

The proof follows from a straightforward argument based on definitions, by induction on the number of elements in a fixed generating set for $M$, which we informally describe:  By definition, $x \in I M$ can  be written, for some positive integer $k$ and elements $r_i \in R$ and $w_i \in M$,  $1 \leq i \leq k$, as   $x = r_1 \cdot w_1 + r_2 \cdot w_2 + \cdots + r_k \cdot w_k$.
What's more, by definition of a finite generating set $u_{1},\dots, u_{n}$ for $M$,  each $w_i$ equals 
$c_{i1} \cdot u_1 + c_{i2} \cdot u_2 + \cdots + c_{in} \cdot u_n$ for appropriate choices of $c_{ij} \in R$. 
Hence, inductively applying associativity, 
\begin{align*}
  x = \sum_{i=1}^{k} r_i  \cdot  w_i  =  \sum_{i=1}^{k} r_i  \cdot  \left(  \sum_{j=1}^n c_{ij}  \cdot  u_j \right) 
  =   \sum_{j=1}^n  \sum_{i=1}^{k} r_i   \cdot  (c_{ij}  \cdot  u_j ) 
=  \sum_{j=1}^n \left( \sum_{i=1}^{k}  c_{ij}  \cdot   r_i \right)   \cdot  u_j.   
\end{align*}
By absorption in ideals, each  $c_{ij}  \cdot r_i$ is in $I$, and since ideals are closed under addition, we inductively conclude that 
the coefficient $a_j := c_{1j} \cdot  r_1 +  c_{2j} \cdot  r_2 +  \cdots c_{kj}\cdot r_k$ of $u_j$ is also in $I$. 

Now we move forward to outline the formal proof of Nakayama's lemma.
We proceed by induction on the number of elements in a generating set for the
$R$-module $M$. 
The base case is the situation when \(M\) requires no generators, so 
$M$ consists solely of the empty scalar combination, i.e., the empty sum, which, by convention, is the zero element.
In other words, $M = 0$ by assumption, and this is also precisely the 
conclusion of Nakayama's Lemma.  Hence the statement trivially, without using the hypothesis that $M = \mathfrak{m} M$.

We turn to the inductive step.  Fixing an arbitrary nonnegative integer $n$, 
we assume that Nakayama's Lemma holds in the case that $M$ has a generating set 
consisting of $n$ elements. 

Now, 
fix an $R$-module $M$ generated by $u_1, \ldots, u_{n+1} \in M$. 
By assumption, \(M = \mathfrak{m} M\), and in particular, 
the generator $u_1$ is an element of the submodule $\mathfrak{m}M$.
The lemma described above guarantees the existence of ring elements $a_1, \ldots, a_n \in \mathfrak{m}$
for which 
\[u_{1} = a_{1} \cdot u_{1} + a_2 \cdot u_2 + \cdots + a_{n+1} \cdot u_{n+1}. \]

Collecting the \(u_{1}\) terms
on the left-hand side of the equation, applying the existence of additive inverses in $R$, and
 distributivity of scalar multiplication for modules, we see that
\begin{equation} \label{equation1: e}
(1 - a_{1}) \cdot u_{1} = a_{2} \cdot u_{2} + \cdots + a_{n+1} \cdot u_{n+1}\text{.}
\end{equation}

Since a proper ideal contains no units and $a_1$ is taken to be in $\mathfrak{m}$, it is not a unit.  
Hence, by the formalized statement mentioned earlier, $1-a_1$ is a unit of $R$.    
Let $b_{1} \in R$ denote its 
multiplicative inverse, so that $b_1 \cdot  (1 - a_{1}) = 1$.
Multiplying this element on the either side of \eqref{equation1: e}, we deduce the following.  
\begin{align*}
b_1 \cdot \left(  (1 - a_{1}) \cdot u_{1} \right)  & = b_1 \cdot  ( a_{2} \cdot u_{2} + \cdots + a_{n+1} \cdot u_{n+1} ) \\
\left( b_1 \cdot  (1 - a_{1})\right) \cdot u_{1}   & = b_1 \cdot  ( a_{2} \cdot u_{2} ) + \cdots + b_1  \cdot   (a_{n+1} \cdot u_{n+1} )  \\
u_1 = 1 \cdot u_{1}   & = ( b_1 \cdot  a_{2}  ) \cdot u_{2}  + \cdots + ( b_1  \cdot  a_{n+1} ) \cdot u_{n+1}  
\end{align*}
In particular, the generator $u_1$ can be written as a scalar combination of the $n$ generators $u_2, \ldots, u_{n+1}$, and is superfluous. Hence $M$ can be generated by $n$ elements, and $M=0$ by the inductive hypothesis. 

\emily[inline]{Emily:  Reflecting, I think it would be very valuable to include the formal proof.  Pieces would work, though it is short, and it could be ``broken up'' among the description of the informal argument.  As is, our description of the proof focuses on the informal argument, which is not new/interesting. }

Showing that
the inductive hypothesis holds in Coq is more work than in this informal
proof, but this extra work is just a lot of bookkeeping.

\bibliographystyle{plain}
\bibliography{references}
\end{document}
