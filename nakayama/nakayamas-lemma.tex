\documentclass[12pt]{amsart}
\usepackage[top=1in, bottom=1in, left=1.2in, right=1.2in ]{geometry}
\usepackage{amsmath,amsfonts,amssymb, amsthm}

\usepackage{libertinus}
\usepackage[T1]{fontenc}

\usepackage{fancyhdr}
\pagestyle{fancy}
\thispagestyle{empty}

\fancyhead[L]{{Lambda Group Project}}
\fancyhead[R]{\thepage}
\fancyhead[C]{\textbf{Nakayama's Lemma}}

\renewcommand{\baselinestretch}{1.2}
\parskip 0.2cm

\newcommand{\HRule}{\begin{center}\rule{\linewidth}{.01cm}\end{center}}

\usepackage[shortlabels]{enumitem}

%THEOREM ENVIRONMENTS
\newtheorem{theorem}{Theorem}
\newtheorem{lemma}[theorem]{Lemma}
\newtheorem{corollary}[theorem]{Corollary}
\newtheorem{proposition}[theorem]{Proposition}

%DEFINITION STYLE ENVIRONMENTS
\theoremstyle{definition}
\newtheorem{definition}[theorem]{Definition}
\newtheorem{notation}[theorem]{Notation}
\newtheorem{example}[theorem]{Example}
\newtheorem{examples}[theorem]{Examples}

%REMARK STYLE ENVIRONMENTS
\theoremstyle{remark}
\newtheorem{remark}[theorem]{Remark}




\begin{document}

\begin{center}
	{\large \textbf{Proof Object:  \emph{Nakayama's Lemma}} } \\ \vspace{.2cm} Statement and Relevant Definitions
\end{center}

\vspace{-.5cm}

\HRule


\begin{definition}[Group]
A \textbf{group} is a set $G$ with a binary operation $\ast$ on $G$ such that the following properties hold.
\begin{enumerate}[topsep=0cm,itemsep=0cm]
\item \textbf{Associativity.}  For all $a, b, c, \in G$, $(a \ast b) \ast c = a \ast (b \ast c)$.
\item \textbf{Identity.}  There exists an element $e \in G$ such that for all $a \in G$, $a \ast e = a$ and $e \ast a = a$.
\item \textbf{Inverses.}  For every $a \in G$ there exists some $b \in G$ for which $a \ast b  = e$ and $b \ast a = e$. 
\end{enumerate}
A group $G$ is called \textbf{abelian} if $a \ast b = b \ast a$ for all $a, b \in G$.
\end{definition}

%\begin{examples}
%The set of integers
%\end{examples}

\begin{definition}[Ring]
A \textbf{ring} is a set $R$ with two binary operations, denoted $+$ and $\cdot$, for which the following hold.
\begin{enumerate}[topsep=0cm,itemsep=0cm]
\item \textbf{Abelian group under addition.}  $R$ is an abelian group under $+$ with identity denoted $0$.
\item \textbf{Associativity of multiplication.}  $(r \cdot s) \cdot t = r \cdot (s \cdot t)$ for all $r, s, t \in R$.
 \item \textbf{Multiplicative identity.}  There is an element  denoted $1$ in $R$ for which $r \cdot 1 = r$ and $1 \cdot r = r$ for every $r \in R$. 
 \item \textbf{Distributivity.} $r \cdot (s + t) = r \cdot s + r \cdot t$ and  $(r + s) \cdot t = r \cdot t + s \cdot t$ for all $r, s, t \in R$.  
\end{enumerate}
A ring $R$ is called \textbf{commutative} if $r \cdot s = s \cdot r$ for all $r, s \in R$. 
\end{definition}


\begin{definition}[Ideal of a commutative ring]
An \textbf{ideal} of a commutative ring $R$ is a subset $I \subseteq R$ for which the following hold. 
\begin{enumerate}[topsep=0cm,itemsep=0cm]
\item \textbf{Abelian group under addition.}  $R$ is an abelian group under $+$ with identity denoted $0$.
\item \textbf{Absorption.} For any $r \in R$ and $a \in I$, the element $r a = a r$ is in $I$. 
\end{enumerate}
An ideal $\mathfrak{m}$ of a ring $R$ is called \textbf{maximal} if it is  a proper ideal, i.e., $\mathfrak{m} \subsetneq R,$ and there is no other proper ideal strictly containing $\mathfrak{m}$, i.e., if $\mathfrak{m} \subseteq J \subsetneq R$ for some ideal $J$ of $R$, then $J = \mathfrak{m}$. 
\end{definition}






\begin{remark}[Characterization of proper ideals] \label{proper-ideal: R}
An ideal $I$ of a commutative ring $R$ satisfies $I = R$ if and only if $1 \in I$:  
If $I=R$, then certainly $1 \in R = I$.  On the other hand, if $1 \in I$, then by the absorption property of ideals, $r = r \cdot 1 \in R$ for any $r \in R$, i.e., $R \subseteq I$, and so $I = R.$  Hence an ideal is proper if and only if it does not contain $1$. 
\end{remark}



\begin{definition}[Multiplicative inverse/unit of a ring] 
An element $s$ of a commutative ring $R$ is a \textbf{(muliplicative) inverse}  of $r \in S$ if $rs = 1$.
If $r\in R$ has a multiplicative inverse, we call it a \textbf{unit} of $R$. 
\end{definition}

\begin{remark}[Multiples of units are again units] \label{multiples-units: R}
Ifotice that if $r \in R$ is not a unit, then neither is $rs$ for any $s \in R$:   If $(rs)t = 1$, then $r$ has inverse $st$ since by the associativity of multiplication in $R$,  $1 = (rs)t = r(st)$. 
\end{remark}



\begin{remark} \label{maximal-ideals-existence: R}
Assuming the axiom of choice, every non-unit $x \in R$ is contained in some maximal ideal:
Given a chain of proper ideals containing $x$
\[
I_1 \subsetneq I_2 \subsetneq I_3 \subsetneq \cdots  \subseteq R
\]
one can check that the union $\bigcup_{i=1}^\infty I_i$ is again an ideal of $R$ containing $x$, so the existence of a maximal ideal follows by Zorn's lemma.  
\end{remark}



\begin{lemma} \label{units-local-ring: L}
Suppose that $R$ is a commutative local ring with maximal ideal $\mathfrak{m}$.  
If $x \in \mathfrak{m}$, then $1-x$ is a unit.
\end{lemma}

\begin{proof}[Informal proof]
Suppose, by way of contradiction, that $1-x$ is not a unit.  Then by Remark \ref{maximal-ideals-existence: R}, $1-x$ is contained in the unique maximal ideal $\mathfrak{m}$.  So then $x$ and $1-x$ are both in $\mathfrak{m}$, and since $-x \in \mathfrak{m}$ as well, $(1+x) + (-x) = 1 \in \mathfrak{m}$. 
This forces $\mathfrak{m} = R$ by Remark \ref{proper-ideal: R}, a contradiction since maximal ideals are proper. Hence $1-x$ is indeed a unit. 
\end{proof}


\begin{definition}
	A \textbf{local ring} is a ring that has a unique maximal ideal. 
\end{definition}


\begin{definition}[Module over a ring]
A \textbf{module} over a ring $R$ is an abelian group $M$ under a binary operation $+$, and a function $\cdot: R \times M \to M$, satisfying the following properties for all $r, s \in R$ and $u, v \in M$.
\begin{enumerate}[topsep=0cm,itemsep=0cm]
\item  $r \cdot (u + v)  = r \cdot u + r \cdot v$.
\item $(r+s) \cdot u = r \cdot u + s \cdot u$.
\item $(rs) \cdot u = r \cdot (s \cdot u)$.
\item $1 \cdot u = u$ for all $u \in M$.
\end{enumerate}
A module is called \textbf{finitely generated} if there exist a fixed finite list $u_1, \ldots, u_n \in M$ such that any 
$w \in M$ can be written as 
$
w = r_1 u_1 + r_1 u_2 + \cdots + r_n u_n
$
for some $r_1, \ldots, r_n \in R$. 
\end{definition}


\begin{definition}
	Suppose that $R$ is a commmutative ring, $I$ is an ideal of $R$, than $M$ is an $R$-module.  Then $IM$ is the set of elements of the form 
	$a_1 u_1 + a_2 u_2 + \cdots + a_k u_k$, where $a_1, \ldots, a_k \in I$ and $u_1, \ldots, u_k \in M$.  
Notice that by the absorption property of ideals, $IM$ is an $R$-module contained in $M$. 	
\end{definition}

Though the following has long been traditionally called a ``lemma,'' it is a fundamental theorem in the field of commutative algebra, and could arguably called the ``Fundamental Theorem of Commutative Algebra.'' 
Below, the notation ``$M=0$'' means that $M$ is the $R$-module consisting of only one element, $0$. 


\begin{theorem}[Nakayama's Lemma]
	Let $M$ be a finitely generated module over a commutative local ring $R$ that has maximal ideal $\mathfrak{m}$. 
	If $M = \mathfrak{m}M$, then $M=0$. 
\end{theorem}

\begin{proof}[Informal Proof]
We apply induction on the number of generators $n$ of $M$. 

For the base case $n=1$, suppose that $u$ generates $M$.	
Then every element of $M$ has the form $ru$ for some $r \in R$, and hence every element of $\mathfrak{m}M$ has the form $a(ru) $
for some $a \in    \mathfrak{m}$. 
By the associative  property of multiplication in $R$, $a(ru) = (ar)u$.  
Then by the absorption property of ideals, $ar \in \mathfrak{m}$, and we conclude that every element of $\mathfrak{m}M$ has the form $xu$ for some element $x \in \mathfrak{m}$ (namely, $x = ar$). 

Since $M = \mathfrak{m} M$ and $u \in M$, we can write $u = xu$ for some $x \in \mathfrak{m}$.  Hence $(1-x) u = 0$.
By Lemma \ref{units-local-ring: L},  $1-x$ is a unit.  Suppose it has inverse $y$,  so \[ 0 = y \cdot 0= y((1-x)u) = (y(1-x)) u = 1 \cdot u = u. \]  Hence every element in $M$ is zero, i.e. $M=0$.

Inductively, fix  $n > 1$ and assume that the only $R$-module $N$ generated by $n-1$ elements that satisfies $N = \mathfrak{m}N$ is $N=0$.

Take an $R$-module $M$ generated by $u_1, \ldots, u_n \in M$. 

Since $M = \mathfrak{m}M$ and $u_1 \in M$, we can write 
\[
u_1 = x_1 u_1 + x_2 u_2 + \cdots + x_n u_n
\]
for some $x_1, \ldots, x_n \in \mathfrak{m}$. 
Then $(1-x_1)u_1 =  x_2 u_2 + \cdots + x_n u_n$, where again by Lemma \ref{units-local-ring: L}, $1-x_1$ is a unit. 
If $z$ is its inverse, then 
\begin{align*}
z ((1-x_1)u_1)  &= z( x_2 u_2 + \cdots + x_n u_n ) \\
(z (1-x_1))u_1 &= z( x_2 u_2 ) + \cdots + z (x_n u_n) \\
u_1 &= (z x_2) u_2 + \cdots + (z x_n) u_n. 
\end{align*}
Hence $M$ is generated by $u_2, \ldots, u_n$, and by the inductive hypothesis, 
$M = 0$.
\end{proof}
\end{document}
