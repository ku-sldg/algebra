\documentclass[12pt]{article}
\usepackage[top=1in, bottom=1in, left=1in, right=1in ]{geometry}
\usepackage{amsmath,amsfonts,amssymb, amsthm}
\usepackage[colorlinks, filecolor=cyan, urlcolor=niceblue]{hyperref}

\usepackage{libertinus}
\usepackage[T1]{fontenc}
\usepackage{fancyhdr}
\usepackage[shortlabels]{enumitem}
\usepackage[mathscr]{euscript}

\renewcommand{\baselinestretch}{1}
\thispagestyle{empty}


\newcommand{\NN}{\mathbb{N}}
\newcommand{\ZZ}{\mathbb{Z}}
\newcommand{\RR}{\mathbb{R}}
\newcommand{\CC}{\mathbb{C}}
\newcommand{\QQ}{\mathbb{Q}}
\newcommand{\Zm}[1]{\mathbb{Z}/{#1}\mathbb{Z}}
\newcommand{\VV}{\mathbb{V}}

\newcommand{\kay}{\mathsf{k}}

\newcommand{\WW}{\mathcal{W}}
\renewcommand{\AA}{\mathcal{A}}
\newcommand{\BB}{\mathcal{B}}

\newcommand{\HRule}{\begin{center}\rule{\linewidth}{.01cm}\end{center}}

\setlength{\parskip}{1em}


%THEOREM ENVIRONMENTS


\newtheorem{theorem}{Theorem}[section]
\newtheorem{lemma}[theorem]{Lemma}
\newtheorem{corollary}[theorem]{Corollary}
\newtheorem{proposition}[theorem]{Proposition}

%DEFINITION STYLE ENVIRONMENTS
\theoremstyle{definition}
\newtheorem{definition}[theorem]{Definition}
\newtheorem*{definition*}{Definition}
\newtheorem{definition-theorem}[theorem]{Definition/Theorem}
\newtheorem{setup}[theorem]{Setup}
\newtheorem*{notation*}{Notation}
\newtheorem{notation}[theorem]{Notation}
\newtheorem{example}[theorem]{Example}
\newtheorem{examples}[theorem]{Examples}


%REMARK STYLE ENVIRONMENTS
\theoremstyle{remark}
\newtheorem{remark}[theorem]{Remark}
\newtheorem*{convention}{Co\usepackage{tikz-cd}nvention}

% http://www.neverendingbooks.org/mumfords-treasure-map

\pagestyle{fancy}
\fancyhead[L]{{\textbf{MATH 831}}}
\fancyhead[R]{\thepage}
\fancyhead[C]{Nakayama's Lemma}



\begin{document}

\begin{center}
	{\large \textbf{Proof Object:  \emph{Nakayama's Lemma}} } \\ \vspace{.2cm} Statement and Relevant Definitions
\end{center}

\vspace{-.5cm}

\HRule


\begin{definition}[Group]
A \textbf{group} is a set $G$ with a binary operation $\ast$ on $G$ such that the following properties hold.
\begin{enumerate}[topsep=0cm,itemsep=0cm]
\item \textbf{Associativity.}  For all $a, b, c, \in G$, $(a \ast b) \ast c = a \ast (b \ast c)$.
\item \textbf{Identity.}  There exists an element $e \in G$ such that for all $a \in G$, $a \ast e = a$ and $e \ast a = a$.
\item \textbf{Inverses.}  For every $a \in G$ there exists some $b \in G$ for which $a \ast b  = e$ and $b \ast a = e$. 
\end{enumerate}
A group $G$ is called \textbf{abelian} if $a \ast b = b \ast a$ for all $a, b \in G$.
\end{definition}

%\begin{examples}
%The set of integers
%\end{examples}

\begin{definition}[Ring]
A \textbf{ring} is a set $R$ with two binary operations, denoted $+$ and $\cdot$, for which the following hold.
\begin{enumerate}[topsep=0cm,itemsep=0cm]
\item \textbf{Abelian group under addition.}  $R$ is an abelian group under $+$ with identity denoted ``$0$.''
\item \textbf{Associativity of multiplication.}  $(r \cdot s) \cdot t = r \cdot (s \cdot t)$ for all $r, s, t \in R$.
 \item \textbf{Multiplicative identity.}  There is an element  denoted ``$1$'' in $R$ for which $r \cdot 1 = r$ and $1 \cdot r = r$ for every $r \in R$. 
 \item \textbf{Distributivity.} $r \cdot (s + t) = r \cdot s + r \cdot t$ and  $(r + s) \cdot t = r \cdot t + s \cdot t$ for all $r, s, t \in R$.  
\end{enumerate}
A ring $R$ is called \textbf{commutative} if $r \cdot s = s \cdot r$ for all $r, s \in R$. 
\end{definition}


\begin{definition}[Ideal of a commutative ring]
An \textbf{ideal} of a commutative ring $R$ is a subset $I \subseteq R$ for which the following hold. 
\begin{enumerate}[topsep=0cm,itemsep=0cm]
\item \textbf{Abelian group under addition.}  $R$ is an abelian group under $+$ with identity denoted ``$0$.''
\item \textbf{Absorption.}  $r a \in I$ for all $r \in R$ and $a \in I$.
\end{enumerate}
An ideal $\mathfrak{m}$ of a ring $R$ is called \textbf{maximal} if it is  a proper ideal, i.e., $\mathfrak{m} \subsetneq R,$ and there is no other proper ideal strictly containing $\mathfrak{m}$, i.e., if $\mathfrak{m} \subseteq J \subsetneq R$ for some ideal $J$ of $R$, then $J = \mathfrak{m}$. 
\end{definition}



\begin{definition}
	A \textbf{local ring} is a ring that has a unique maximal ideal. 
\end{definition}

\begin{remark}
Assuming the axiom of choice, every ring has at least one maximal ideal since
given ideals
\[
I_1 \subsetneq I_2 \subsetneq I_3 \subsetneq \cdots R
\]
one can check that the union $\bigcup_{i=1}^\infty I_i$ is again an ideal of $R$, so the existence of a maximal ideal follows by Zorn's lemma.  In fact, any \emph{unit} of $R$--element with no multiplicative inverse--is contained in a maximal ideal. 
\end{remark}



\begin{definition}[Module over a ring]
A \textbf{module} over a ring $R$ is an abelian group $M$ under a binary operation $+$, and a function $\cdot: R \times M \to M$, satisfying the following properties for all $r, s \in R$ and $u, v \in M$.
\begin{enumerate}[topsep=0cm,itemsep=0cm]
\item  $r \cdot (u + v)  = r \cdot u + r \cdot v$.
\item $(r+s) \cdot u = r \cdot u + s \cdot u$.
\item $(rs) \cdot u = r \cdot (s \cdot u)$.
\item $1 \cdot u = u$ for all $u \in M$.
\end{enumerate}
A module is called \textbf{finitely generated} if there exist a fixed finite list $u_1, \ldots, u_n \in M$ such that any 
$w \in M$ can be written as 
$
w = r_1 u_1 + r_1 u_2 + \cdots + r_n u_n
$
for some $r_1, \ldots, r_n \in R$. 
\end{definition}

Though the following has long been traditionally called a ``lemma,'' it is a fundamental theorem in the field of commutative algebra, and could arguably called the ``Fundamental Theorem of Commutative Algebra.''

\begin{definition}
	Suppose that $R$ is a commmutative ring, $I$ is an ideal of $R$, than $M$ is an $R$-module.  Then $IM$ is the set of elements of the form 
	$a_1 u_1 + a_2 u_2 + \cdots + a_k u_k$, where $a_1, \ldots, a_2$.  
Due to the absorption property of ideals, $IM$ is an $R$-module contained in $M$. 	
\end{definition}


\begin{theorem}[Nakayama's Lemma]
	Let $M$ be a finitely generated module over a local ring $R$ that has maximal ideal $\mathfrak m$. 
	If $M = \mathfrak{m}M$, then $M=0$. 
\end{theorem}

\begin{proof}[Informal Proof]
We apply induction on the number of generators $n$ of $M$. 
For the base case $n=1$, suppose that $u$ generates $M$.	
Then every element of $M$ has the form $ru$ for some $r \in R$, and hence every element of $\mathfrak{m}M$ has the form $a(ru) = (ar)u = (ra)u$ for some $a \in    \mathfrak{m}$.  By the absorption property of ideals, $ra \in \mathfrak{m}$, and we conclude that every element of $\mathfrak{m}M$ has the form $xu$ for some $x \in \mathfrak{m}$. 

Suppose that $M = \mathfrak{m} M$.  Then since $u \in M$, $u = xu$ for some $x \in \mathfrak{m}$.  Hence $(1-x) u = 0$. We claim that $1-x$ has a multiplicative inverse. If not, then it is contained in $\mathfrak{m}$, but then since $x \in m$, we have that $1 = (1-x)+x \in \mathfrak{m}$, contradicting the fact that $\mathfrak{m}$ is a proper ideal of $R$.  Hence $1-x$ has a multiplicative inverse $y$, and so \[ 0 = y \cdot 0= y((1-x)u) = (y(1-x)) u = 1 \cdot u = u. \]  This forces $M=0$, and the statement holds. 

Now, inductively, for some $n \geq 1$,  assume that $M$ is generated by $u_1, \ldots, u_n \in M$.  Let $Ru_1$ denote the $R$-module generated by $u_1$, and let 
 if $N = M/R u_1$ be the $R$-module consisting of equivalence classes of elements of $M$ with respect to the equivalence relation given by $w \sim w'$ if and only if $w - w' \in R u_1$.  Then $N$ is generated by the equivalence classes of $u_2, \ldots, u_n$, $(n-1)$-many element of $N$, and $\mathfrak{m}N = N$ still holds.  
 So by the inductive hypothesis, $N=0$, which says that $M = R u_1$.  But we have already done the case where $n=1$. 
\end{proof}
\end{document}
